\documentclass[10pt, onecolumn]{IEEEtran}

\usepackage{etoolbox}
\usepackage{scrextend}

% remove \centering from \section
\patchcmd{\section}{\centering}{}{}{}

\begin{document}
\title{An Analysis of Risk\\ in Open-Source Projects}

\author{\IEEEauthorblockN{Róisín Ní Bhriain}
\IEEEauthorblockA{\\Email: roisin.nibhriain3@mail.dcu.ie\\
                    Student Number: 23269640\\}
\and
\IEEEauthorblockA{Sneha Dechamma Mallengada Suresh\\
    Email: sneha.mallengadasuresh3@mail.dcu.ie\\
    Student Number: 23262168\\}}


\maketitle

\date{2024-07-23}


\begin{description}
    \item[]
    \begin{center}
        \item[\textbf{Technical Manual - 23-07-2024}] 
        \end{center}
    \item[]
\end{description}



\begin{abstract}
\begin{addmargin}[4em]{4em}
This paper explores the methods used in the prediction of risk when using open-source software. There are a number of methods used and the aim is to combine these to give a more comprehensive view of the risks in using a particular open-source project. We examine the literature to decide on an appropriate prediction algorithm for each section. We provide a graph of the final results which can be used to aid in finding where the risk lies in the dependency trees. 
\end{addmargin}
\end{abstract}


\section{Introduction}


\section{Related Work}

\section{Datasets}

\section{Method}
This section explores the prediction of risk methods that we used.

\subsection{Finding Dependencies Algorithm}
To find dependencies of different projects we focused Maven dependency trees. 

\subsection{Project Meta-Data Prediction}

\subsection{Vulnerability Database Prediction}

\section{Evaluation}

\section{Results}


\section{Conclusions \& Further Work}
The conclusion goes here.

\begin{thebibliography}{1}

\bibitem{IEEEhowto:kopka}
H.~Kopka and P.~W. Daly, \emph{A Guide to \LaTeX}, 3rd~ed.\hskip 1em plus
  0.5em minus 0.4em\relax Harlow, England: Addison-Wesley, 1999.

\end{thebibliography}

\appendices
\section{Graphs}
This section contains other important graphs that we did not include in the main body of the project.


\ifCLASSOPTIONcaptionsoff
  \newpage
\fi

\end{document}


